\documentclass[]{article}
\usepackage{lmodern}
\usepackage{amssymb,amsmath}
\usepackage{ifxetex,ifluatex}
\usepackage{fixltx2e} % provides \textsubscript
\ifnum 0\ifxetex 1\fi\ifluatex 1\fi=0 % if pdftex
  \usepackage[T1]{fontenc}
  \usepackage[utf8]{inputenc}
\else % if luatex or xelatex
  \ifxetex
    \usepackage{mathspec}
  \else
    \usepackage{fontspec}
  \fi
  \defaultfontfeatures{Ligatures=TeX,Scale=MatchLowercase}
\fi
% use upquote if available, for straight quotes in verbatim environments
\IfFileExists{upquote.sty}{\usepackage{upquote}}{}
% use microtype if available
\IfFileExists{microtype.sty}{%
\usepackage{microtype}
\UseMicrotypeSet[protrusion]{basicmath} % disable protrusion for tt fonts
}{}
\usepackage[margin=1in]{geometry}
\usepackage{hyperref}
\hypersetup{unicode=true,
            pdfborder={0 0 0},
            breaklinks=true}
\urlstyle{same}  % don't use monospace font for urls
\usepackage{graphicx,grffile}
\makeatletter
\def\maxwidth{\ifdim\Gin@nat@width>\linewidth\linewidth\else\Gin@nat@width\fi}
\def\maxheight{\ifdim\Gin@nat@height>\textheight\textheight\else\Gin@nat@height\fi}
\makeatother
% Scale images if necessary, so that they will not overflow the page
% margins by default, and it is still possible to overwrite the defaults
% using explicit options in \includegraphics[width, height, ...]{}
\setkeys{Gin}{width=\maxwidth,height=\maxheight,keepaspectratio}
\IfFileExists{parskip.sty}{%
\usepackage{parskip}
}{% else
\setlength{\parindent}{0pt}
\setlength{\parskip}{6pt plus 2pt minus 1pt}
}
\setlength{\emergencystretch}{3em}  % prevent overfull lines
\providecommand{\tightlist}{%
  \setlength{\itemsep}{0pt}\setlength{\parskip}{0pt}}
\setcounter{secnumdepth}{0}
% Redefines (sub)paragraphs to behave more like sections
\ifx\paragraph\undefined\else
\let\oldparagraph\paragraph
\renewcommand{\paragraph}[1]{\oldparagraph{#1}\mbox{}}
\fi
\ifx\subparagraph\undefined\else
\let\oldsubparagraph\subparagraph
\renewcommand{\subparagraph}[1]{\oldsubparagraph{#1}\mbox{}}
\fi

%%% Use protect on footnotes to avoid problems with footnotes in titles
\let\rmarkdownfootnote\footnote%
\def\footnote{\protect\rmarkdownfootnote}

%%% Change title format to be more compact
\usepackage{titling}

% Create subtitle command for use in maketitle
\providecommand{\subtitle}[1]{
  \posttitle{
    \begin{center}\large#1\end{center}
    }
}

\setlength{\droptitle}{-2em}

  \title{}
    \pretitle{\vspace{\droptitle}}
  \posttitle{}
    \author{}
    \preauthor{}\postauthor{}
    \date{}
    \predate{}\postdate{}
  

\begin{document}

\section{Communications
interpersonnelles}\label{communications-interpersonnelles}

\subsection{Difficultés rencontrées}\label{difficultes-rencontrees}

\subsubsection{R vs python}\label{r-vs-python}

La principale difficulté rencontrée au début est de passer de
l'apprentissage du langage de programmation \emph{python} à \emph{R}. Ce
sont tous les deux des langages de programmation interprétés.
\emph{python} a été créé pour faire de la programmation informatique
généraliste, il est utilisé dans de larges domaines par des
informaticiens. A l'inverse, R est dédié aux analyses statistiques,
plutôt utilisées par des spécialistes ou des scientifiques.

Dans le domaine du \emph{data scientist}, \emph{R} et \emph{python} sont
courament employé.

\subsubsection{Shiny communication entre ui.R et
server.R}\label{shiny-communication-entre-ui.r-et-server.r}

Les application web géré par shiny utilise deux fonctions communiquant
entre elles \textbf{ui} et le \textbf{server}

Le schéma de communication basique entre les deux scripts commence par
la déclaration d'une variable \emph{inputId = ma\_variable} dans ui.R.
Celui-ci est appelé dans server.R sous la forme input\$ma\_variable,
cette variable sera ensuite traitée dans un bloc de code délimiter par
des crochets.

Shiny utilise du Javascript pour dynamiser l'interface de l'utilisateur
sous une couche de code masqué, cette couche simplifie grandement le
travail avec R. Si on sort du cadre de l'utilisation prévu par Shiny, on
se heurte à de grands soucis de codage. Shiny restreint donc, la
communication entre les différents blocs de code. Dans certaines
situations cela complique le travail.

\subsubsection{Apport au sein de
l'entreprise}\label{apport-au-sein-de-lentreprise}

Pour l'instant, la contribution à l'entreprise revient principalement
aux relevés des boutures (monitoring) et l'encodage, ceci permet de
libérer du temps au technicien. L'application qui est en développement
est utilisé dans le travail de mémoire de Madeleine Gilles.


\end{document}
